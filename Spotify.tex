\documentclass[40pt]{article}
\usepackage{babel}
\usepackage[T1]{fontenc}
\usepackage{textcomp}
\usepackage[utf8]{inputenc} % Puede depender del sistema o editor
\usepackage{enumerate}


\title{\textbf{Universidad Veracruzana} }
\date{\textbf{Facultad de Negocios y Tecnologias} }

\begin{document}
\maketitle
%\section{Integrantes}
\textsf{\Large 
\\
\\
Experiencia Educativa: Base De Datos No Convencional. \\}
\\
\\ 
\maketitle
%\section{Integrantes}
\textsf{\Large Catedratico: Centeno Tellez Adolfo. \\}
\\
\\
\maketitle
%\section{Integrantes}
\textsf{\Large Tema: Proyecto Spotify. \\}
\\
\\
\maketitle
%\section{Integrantes}
\textsf{\Large Alumno: Zarate Espinosa Jose Pedro. \\}
\\
\\ 
\maketitle
%\section{Integrantes}
\textsf{\Large Grupo: 601 ISW 1° Parcial \\}
\\
\\
\maketitle
%\section{Integrantes}
\textsf{\Large Fecha de Entrega: 06 de Abril del 2021 \\}

\newpage

\maketitle
%\section{Integrantes}
\textsf{\ \\
\textbf{Introducción:}\\
\\
El proyecto que se realizo consta de un Spotify clon, el cual se representó mediante react de acuerdo con lo aprendido este va implementado de igual manera con Firebase ya que es donde se guardara las imágenes, titulo, descripción.
\\
\\
Estaremos ocupando Firebase ya que no es SQL tal y como lo marca el programa, de igual forma se utilizara Facebook For Developers, todo esto con la finalidad de implementar lo aprendido en la E.E de Bases De Datos No Convencionales.
\\}

\maketitle
%\section{Integrantes}
\textsf{\ \\
\textbf{Desarrollo:}\\
Para el desarrollo del proyecto se utilizará react este se usará para el prototipado del Spotify, como primera parte se realizo un login con Facebook For Developers donde tuvimos que darnos de alta con nuestro Facebook personal de esta manera pudimos ocupar la herramienta de ingresar con el login perteneciente a esta plataforma.
\\
\\
Como segunda parte se realizo un apartado de Playlist donde el usuario podrá ver su lista de reproducción la cual se encuentra semi centrada y aun costado el título y debajo del título la descripción de la canción, por último, saldrá el módulo de reproducción el cual contiene la música, el usuario tendrá un CRUD básico por debajo de lo antes mencionado.
\\
\\
En la pestaña de agregar Playlist es algo muy intuitiva la cual muestra dos campos donde se ingresará el título de la canción y por debajo la descripción, a esta se le agrego dos botones, en el primero se selecciona el archivo que se desea subir a la base de datos el cual puede ser una imagen, PDF, y el que mas nos interesa video, después de seleccionarlo por debajo está el segundo botón el cual sube el archivo a la base de datos.
\\
\\
Como observación primero se debe subir el documento seleccionado es decir el video y posteriormente por debajo aparecerá en link del video, así como el módulo después de visualizarlo se dará enviar el título y la descripción, para después poder visualizarlo en la lista de reproducción.
\\
\\
Si no se realiza como ya antes se menciono el url no se guardará dentro del título y descripción, se menciona ya que este fue un pequeño detalle que nos ocasiono problemas debido a que subíamos el video y enviábamos los datos al mismo tiempo, y como el video es pesado dilata en subirse a la base de datos y se guardaba vacía la parte de la url, aunque se subía.
\\
\\
Por ultima parte una vez que funciono el programa se agrego el login de Facebook el cual aparece como primera pantalla para ingresar, debes aceptar continuar con Facebook para poder ingresar a la lista de reproducción.
\\}

\maketitle
%\section{Integrantes}
\textsf{\ \\
\textbf{Conclusión:}
\\
\\
Para concluir cabe destacar que el objetivo principal si fue favorable en este proyecto ya que cumple con lo requerido por parte del profesor y cumple las funciones básicas de un Spotify, se trabajo varias horas en problemas pequeños pero muy significativos uno de ellos era que no se mostraba el modulo de reproducción en la lista de Playlist aunque no era un problema de código si no de el documento se subiera primero de igual forma enviar, pero finalmente se cloncluyo el proyecto satisfactoriamente.
\\}



\end{document}
